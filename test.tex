\documentclass[12pt]{article}

\usepackage[top=50pt,bottom=50pt,left=50pt,right=50pt]{geometry}

\usepackage{listings}
\usepackage{color}

\definecolor{mygreen}{rgb}{0,0.6,0}
\definecolor{mygray}{rgb}{0.5,0.5,0.5}
\definecolor{mymauve}{rgb}{0.58,0,0.82}

\lstset{
  basicstyle=\linespread{0.85}\ttfamily\footnotesize,        % the size of the fonts that are used for the code
  breakatwhitespace=false,         % sets if automatic breaks should only happen at whitespace
  breaklines=true,                 % sets automatic line breaking
  captionpos=b,                    % sets the caption-position to bottom
  commentstyle=\color{mygreen},    % comment style
  deletekeywords={...},            % if you want to delete keywords from the given language
  escapeinside={\%*}{*)},          % if you want to add LaTeX within your code
  extendedchars=true,              % lets you use non-ASCII characters; for 8-bits encodings only, does not work with UTF-8
  keepspaces=true,                 % keeps spaces in text, useful for keeping indentation of code (possibly needs columns=flexible)
  keywordstyle=\color{blue},       % keyword style
  language=Python,                 % the language of the code
  otherkeywords={*,...},            % if you want to add more keywords to the set
  numbers=left,                    % where to put the line-numbers; possible values are (none, left, right)
  numbersep=5pt,                   % how far the line-numbers are from the code
  numberstyle=\tiny\color{black}, % the style that is used for the line-numbers
  rulecolor=\color{black},         % if not set, the frame-color may be changed on line-breaks within not-black text (e.g. comments (green here))
  showspaces=false,                % show spaces everywhere adding particular underscores; it overrides 'showstringspaces'
  showstringspaces=false,          % underline spaces within strings only
  showtabs=false,                  % show tabs within strings adding particular underscores
  stepnumber=1,                    % the step between two line-numbers. If it's 1, each line will be numbered
  stringstyle=\color{mymauve},     % string literal style
  tabsize=2,	                   % sets default tabsize to 2 spaces
  numbers=none,
}

\PassOptionsToPackage{hyphens}{url}
\usepackage[colorlinks = true,linkcolor = blue]{hyperref}
\usepackage{times}


\begin{document}


\title{TP: integrating github project with travis}
\author{Tommaso Vinci}
\date{\today}
\maketitle


\section*{Create repository}

test 

Here \texttt{UserName} is your Github username.

\subsection*{From scratch}

Go to \url{https://github.com/UserName?tab=repositories} Click on \emph{New}

In the following page fill the \emph{Repository name} (here \texttt{test}) and optionally the \emph{Description}. Let the repo public and click on \emph{Create repository}.

Then in a terminal:

\begin{lstlisting}[language=Bash] 
git clone git@github.com:UserName/test.git
git add test.tex README.md .travis.yml
git commit -m "First commit"
git push -u origin master
\end{lstlisting}

The last command will ask for your Github credentials.

\subsection*{From my repo}
Sign in to Github and go to \url{https://github.com/UserName/test.}




\section*{Sync repo with Travis}

Go to \url{https://travis-ci.org/profile/UserName} and click on \emph{Sync account}

below an \emph{UserName/test} should appear: toggle it (should become green).

got to \url{https://github.com/settings/tokens} click on \texttt{Generate new token}. Type in your password

Type \texttt{deployToken} in the \emph{Token description}. Ceck the \emph{repo} scope.

Click on \emph{Generate token}. In the following page copy in the clipboard the generated string (this will be the only chance you have to copy).

Go to \url{https://travis-ci.org/UserName/test/settings}. 

Check the \emph{Build only if .travis.yml is present}.

In the \emph{Environment Variables} section, type in a name for the variable (\emph{e.g.} \texttt{deployKey}) and paste in the \emph{Value} field the string you copied before on Github. Click on \emph{Add}.

Now every push will trigger a build and if the commit is a tagged commit, the generated pdf will be uploaded to \url{https://github.com/UserName/test/releases}.

To tag a commit:
\begin{lstlisting}[language=Bash] 
git tag -a v1.0 -m "my version 1.0"
git push origin v1.0
\end{lstlisting}

To move the current tag to the current commit (to trigger a deploy):
\begin{lstlisting}[language=Bash] 
git push origin :refs/tags/v1.0
git tag -fa v1.0
git push origin master --tags
\end{lstlisting}

\section*{File \texttt{.travis.yml}}

\lstinputlisting{.travis.yml}



\end{document}
